% This file is part of Derpnet.

% Derpnet is free software: you can redistribute it and/or modify it under the
% terms of the GNU General Public License as published by the Free Software
% Foundation, either version 3 of the License, or (at your option) any later
% version.

% Derpnet is distributed in the hope that it will be useful, but WITHOUT ANY
% WARRANTY; without even the implied warranty of MERCHANTABILITY or FITNESS
% FOR A PARTICULAR PURPOSE.  See the GNU General Public License for more
% details.

% You should have received a copy of the GNU General Public License along with
% Derpnet.  If not, see <http://www.gnu.org/licenses/>.
 
% Copyright (C) 2011 The Derpnet Team.

%%%
%
% Header author: George Silvis, III
% Purpose: Generic header file for all of the mathematical documents I make.
% Nota Bene: Not all of the text in this is mine.
% Last Modified: 01/22/11
%
%%%

\documentclass[11pt]{article}
\usepackage[utf8]{inputenc}

\usepackage{geometry}
\geometry{letterpaper}
%   read geometry.pdf for detailed page layout information

\usepackage{graphicx}

\usepackage[parfill]{parskip} % Activate to begin paragraphs with an empty
                              % line rather than an indent

%%% PACKAGES
\usepackage{booktabs} % for much better looking tables
\usepackage{array} % for better arrays (eg matrices) in maths
\usepackage{paralist} % very flexible & customisable lists
                      % (eg. enumerate/itemize, etc.)
\usepackage{verbatim} % adds environment for commenting out blocks of text &
                      % for better verbatim
\usepackage{subfig} % make it possible to include more than one captioned
                    % figure/table in a single float

\usepackage{amsmath}
\usepackage{amssymb}
\usepackage{amsthm}

\usepackage{hyperref} % for urls

%%% Commands I've created:

\newcommand{\diff}[1]{\mathrm{d}#1}
\newcommand{\ptdf}[1]{\partial#1}
\newcommand{\dee}[2]{\diff{#1}/\diff{#2}}
\newcommand{\Dee}[2]{\frac{\diff{#1}}{\diff{#2}}}
\newcommand{\dde}[2]{\ptdf{#1}/\ptdf{#2}}
\newcommand{\Dde}[2]{\frac{\ptdf{#1}}{\ptdf{#2}}}
\newcommand{\tends}[1]{\stackrel{#1 \to \infty}{\longrightarrow}}
\newcommand{\nat}[1]{#1 \in \mathbb{N}}
\newcommand{\real}[1]{#1 \in \mathbb{R}}
\newcommand{\z}[1]{#1 \in \mathbb{Z}}
\newcommand{\del}{\mathbf{\nabla}}
\newcommand{\FFF}[2]{\left< #1,#2 \right>}
\newcommand{\SFF}[2]{\left<\left< #1,#2 \right>\right>}
\newcommand{\mx}[4]{\begin{bmatrix} #1 & #2 \\ #3 & #4 \end{bmatrix}}
\newcommand{\jac}[4]{\matrix{\dde{#1}{#3}}{\dde{#1}{#4}}{\dde{#2}{#3}}{\dde{#2}{#4}}}
\newcommand{\JAC}[4]{\matrix{\Dde{#1}{#3}}{\Dde{#1}{#4}}{\Dde{#2}{#3}}{\Dde{#2}{#4}}}
\newcommand{\caj}[4]{\matrix{\dde{#1}{#3}}{\dde{#2}{#3}}{\dde{#1}{#4}}{\dde{#2}{#4}}}
\newcommand{\CAJ}[4]{\matrix{\Dde{#1}{#3}}{\Dde{#2}{#3}}{\Dde{#1}{#4}}{\Dde{#2}{#4}}}
\newcommand{\ev}[1]{\text{E} \left\lbrack #1 \right\rbrack}
\newcommand{\var}[1]{\text{Var} \left\lbrack #1 \right\rbrack}
\newcommand{\mgf}[1]{\ev{e^{t#1}}}
\DeclareMathOperator{\sech}{sech} % Why didn't these exist?
\DeclareMathOperator{\csch}{csch}

%%% SECTION TITLE APPEARANCE
\usepackage{sectsty}
\allsectionsfont{\sffamily\mdseries\upshape} % (See the fntguide.pdf for font help)
% (This matches ConTeXt defaults)

%%% ToC (table of contents) APPEARANCE
\usepackage[nottoc,notlof,notlot]{tocbibind} % Put the bibliography in the ToC
\usepackage[titles,subfigure]{tocloft} % Alter the style of the Table of Contents
\renewcommand{\cftsecfont}{\rmfamily\mdseries\upshape}
\renewcommand{\cftsecpagefont}{\rmfamily\mdseries\upshape} % No bold!

\usepackage{fancyhdr}
\pagestyle{fancy} % options: empty , plain , fancy
\renewcommand{\headrulewidth}{0.4pt}
\setlength{\headheight}{26pt}

\lhead{Derpnet \\ Technical Specification}
\chead{}
\rhead{[internal] \\ Revision 0}

\lfoot{}
\cfoot{\thepage}
\rfoot{}

%%%
%
% End header.
%
%%%

\title{Official Technical Specifications}
\author{The Derpnet Team}
%\date{}

\begin{document}
\maketitle

\section{Preamble}

This is the documentation of standards for the Derpnet secure communication
setup.  The latest version of this document can be found in our git repository
which is located at
\href{https://github.com/BUILDS-/Derpnet}{https://github.com/BUILDS-/Derpnet}.

\subsection{Licensing}

% it would be really awesome (not to mention really useless) to make this a
% partial quine so that we only need to put the text of the GPL in once

Derpnet is free software: you can redistribute it and/or modify it under the
terms of the GNU General Public License as published by the Free Software
Foundation, either version 3 of the License, or (at your option) any later
version.

Derpnet is distributed in the hope that it will be useful, but WITHOUT ANY
WARRANTY; without even the implied warranty of MERCHANTABILITY or FITNESS FOR
A PARTICULAR PURPOSE.  See the GNU General Public License for more details.

You should have received a copy of the GNU General Public License along with
Derpnet.  If not, see $<$\href{http://www.gnu.org/licenses/}{http://www.gnu.org/licenses/}$>$.
 
Copyright (C) 2011 The Derpnet Team.

\subsection{Contributions and External Code}

In accordance with our license, we will not depend on any proprietary code.
While we prefer the GPL to other licenses, we recognize that this is not ideal
for all developers and contributors.  In order to not restrict contributions
and existing code use, we will accept and use code under BSD, MIT, and other
open source licenses, though for ease of distribution, this code may
eventually be replaced with GPL-licensed code.

\subsection{Declaration of Purpose}

The Derpnet project is designed with three goals in mind: security,
transparency, and usability.  We believe that all three are necessary, and
that any system missing even one of the three is divergent from ideal.

\subsubsection{Security}

For us, ``security'' encompasses both security in the traditional sense as
well as anonymity.  Traditionally, security has often come at the cost of
anonymity.  For example, email accounts have often been protected by
``security questions'' requiring personal information to answer.  With this
project, however, we are exploring the idea that this dependence on personal
information from the ``real world'' which many would prefer to keep private
may be unnecessary.  Instead, one can create an identity completely separate
from who they were born as, and attach everything to that, rather than to
their name.

For example, consider a service providing secure email.  This particular
service performs all of its communication with clients over ssl/tls and the
https protocol for web interfacing.  To register an account, one need only
entire the desired account name and password of said account, and it will be
granted.  The user can then use this account to register for other services
based off of the email account thus created.  Anonymity is protected, as there
is no way to determine who the user is from the information provided to the
email system (with the exception of IP-based analysis, which can be easily
avoided if the user connected over Tor or similar service).  Security in the
traditional sense can also be supplied, as the email account that has been
created can be used as the basis for authentication of other services.  The
user need not even trust the email provider with the text of their emails;
with the ready availability of technologies such as PGP, this is no longer
necessary.

The one down-side to this approach is that the email becomes a choke-point.
That is, if the user forgets the passphrase to the email account, then,
depending on the setup of the services that have been anchored upon it, the
identity may be completely lost.  Fortunately, many services today allow users
to easily change the email basis for their identity, which does much to
relieve this problem.

\subsubsection{Transparency}

For the threat vector that Derpnet is designed to protect against, security
through obscurity is no security at all.  An attacker is capable of
reverse-engineering any component of a system, and many are skilled enough or
have resources enough that the time to reverse a system is negligible.  By
this choice of attacker to defend against, we do not deny the problem of
lower-tech attacks such as keyboard logging for the purpose of passphrase
sniffing, but rather suggest that defenses against such attacks as these be
employed in addition to the protection a service like Derpnet provides.

Since there is thus nothing to be gained by the concealment of our system
architecture and implementation, Derpnet is Open-Source Software under the
terms of the GNU GPL.  This means that any user is free to download the source
code and tinker with it.  The purpose of our core team is to function as both
development leads and a semi-regulatory body so that our code remains cohesive
enough to be useful while adding new features, bug-fixes, and the like.

\subsubsection{Usability}

We judge any system that cannot be used, no matter how secure or transparent,
to be without purpose.  It is our task as developers to ensure that our system
is not only useful to but also usable by all of our users.  To that end, code
integrity is our responsibility, as well as adherence to standards.  To
express that in practical terms, if we claim code should compile on your
setup, and it does not, it is our fault, not yours.  In addition, we have
provided multiple means of access to all of our subsystems, as will be
discussed below.

\section{Architecture}

In accordance with our attempt toward transparency, this section attempts to
explain the infrastructure that will power the initial Derpnet.  We make use
of the anonymity-model discussed in the security section.

\subsection{Email Server}

We intend to use an email service as the basis for all of the other services
we provide.  There will be multiple ways to access as an email account: a
secure (https) webmail interface will exist, and additionally we will provide
secure pop and / or imap access with smtp message delivery.  Creating an
account on this system will require no other information than the desired
username and passkey, though other procedures to eliminate abuse (such as
idle account deletion, or the dreaded CAPTCHA) may require implementation and
use as well.

To encourage proper use of services, we encourage users to make use of the
email thus created as the basis for all other services, though, in the
interests of both usability and security, we will not require it.

\subsection{The IRC Protocol}

Internet Relay Chat is one of the oldest forms of internet-based inter-user
communication, and is unique for its room-based architecture.  As it has been
in use for many years, modifications to the basic protocol have developed,
some useful, and some less so.  Perhaps the most important has been the
implementation of ssl connections (typically on port 6697).  In addition, many
servers provide NickServ implementations which allow users to authenticate
themselves, which increases security without sacrificing any more anonymity
than an email address.  However, with the increased use of Tor and other
proxies, many networks have felt the need to block the use of such services in
order to prevent their abuse for DDoS and other attacks.

Obviously, if it were possible to use Tor and other anonymizing networks to
connect to an IRC server without possibility of their abuse, it would be
beneficial from a security perspective.  We believe that we can use NickServ
to solve this problem.  By requiring all users to be registered with NickServ
(depositing an email for confirmation purposes), there becomes no inherent
difference between a Tor / proxied connection and any other connection.  Sure,
the user must identify, but we think, after our own experience with IRC, that
most serious users will want to do this anyway, so the inconvenience is
minimal.

In addition, we define three methods of connection to IRC: legacy, encrypted,
and secure.

\subsubsection{Legacy IRC Connection}

This method will attempt to comply with the IRC RFC documents as much as is
possible without compromise to the security of the servers themselves.  We
will include the NickServ implementation discussed above in this mode of
connection.  This ``legacy'' will operate on port 6667, which has become the
\textit{de facto} standard for unencrypted IRC.  This will only be a temporary
means of IRC connection, and, as it is considered insecure, will be depricated
upon readiness of the ``encrypted'' connection schema and removed entirely
upon readiness of the ``secure'' connection schema.

\subsubsection{Encrypted IRC Connection}

This mode of connection will operate off 6697, which has become standard, as
well as potentially other ports.  This mode of interface is designed to allow
existing clients, such as irssi, pidgin/Adium/finch, and others that have
given security at least a token amount of work, to be able to use the Derpnet
infrastructure.  All connections will be encrypted ssl, and will attempt to
provide as much of the ``secure'' connection schema as is possible without
breaking compatability with these clients.

\subsubsection{Secure IRC Connection}

This is our most secure connection scheme.  All connections to and from the
server will be encrypted using OTR.  In addition, to both increase security
and reduce server load, where possible the two participants of a private query
will be directly connected over OTR.  We will again use the NickServ scheme
mentioned above to prevent DDoS.

Finally, we will introduce a new mode for channels similar to +R (which
indicates that only users registered users can join) such that only users that
are connected over this secure method may join.  (TODO: decide what mode this
should be.  We could commandeer +R since only registered users can exist
on-network anyway, or make a new one to eliminate confusion.)

\section{Unresolved Questions}

In the order that they occurred to me in:

\begin{itemize}
\item 
  What should the new mode be?
\item
  What of the channel prefixes from the RFC do we want to implement?
\item
  Are we C++ exclusive, or do we allow language mixing?
\end{itemize}

\section{Further Reading}

We have included the IRC RFCs in our repository.  We also recommend perusal of
any documents you can find on OTR, as well as spending some time on an IRC
network to get a feel for how the system as a whole behaves.  In addition, any
expertise / insight into email systems would be greatly appreciated.

\section{Credits}

Derpnet is: $\alpha, \beta, \gamma, \delta$.

\end{document}
