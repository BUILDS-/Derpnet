% This file is part of Derpnet.
%
% Derpnet is free software: you can redistribute it and/or modify
% it under the terms of the GNU General Public License as published by
% the Free Software Foundation, either version 3 of the License, or
% (at your option) any later version.
%
% Derpnet is distributed in the hope that it will be useful,
% but WITHOUT ANY WARRANTY; without even the implied warranty of
% MERCHANTABILITY or FITNESS FOR A PARTICULAR PURPOSE.  See the
% GNU General Public License for more details.
%
% You should have received a copy of the GNU General Public License
% along with Derpnet.  If not, see <http://www.gnu.org/licenses/>.
% 
% Copyright (C) 2011 The Derpnet Team.

%%%
%
% Header author: George Silvis, III
% Purpose: Generic header file for all of the mathematical documents I make.
% Nota Bene: Not all of the text in this is mine.
% Last Modified: 01/22/11
%
%%%

\documentclass[11pt]{article}
\usepackage[utf8]{inputenc}

\usepackage{geometry}
\geometry{letterpaper}
%   read geometry.pdf for detailed page layout information

\usepackage{graphicx}

\usepackage[parfill]{parskip} % Activate to begin paragraphs with an empty line rather than an indent

%%% PACKAGES
\usepackage{booktabs} % for much better looking tables
\usepackage{array} % for better arrays (eg matrices) in maths
\usepackage{paralist} % very flexible & customisable lists (eg. enumerate/itemize, etc.)
\usepackage{verbatim} % adds environment for commenting out blocks of text & for better verbatim
\usepackage{subfig} % make it possible to include more than one captioned figure/table in a single float

\usepackage{amsmath}
\usepackage{amssymb}
\usepackage{amsthm}

\usepackage{hyperref} % for urls

%%% Commands I've created:

\newcommand{\diff}[1]{\mathrm{d}#1}
\newcommand{\ptdf}[1]{\partial#1}
\newcommand{\dee}[2]{\diff{#1}/\diff{#2}}
\newcommand{\Dee}[2]{\frac{\diff{#1}}{\diff{#2}}}
\newcommand{\dde}[2]{\ptdf{#1}/\ptdf{#2}}
\newcommand{\Dde}[2]{\frac{\ptdf{#1}}{\ptdf{#2}}}
\newcommand{\tends}[1]{\stackrel{#1 \to \infty}{\longrightarrow}}
\newcommand{\nat}[1]{#1 \in \mathbb{N}}
\newcommand{\real}[1]{#1 \in \mathbb{R}}
\newcommand{\z}[1]{#1 \in \mathbb{Z}}
\newcommand{\del}{\mathbf{\nabla}}
\newcommand{\FFF}[2]{\left< #1,#2 \right>}
\newcommand{\SFF}[2]{\left<\left< #1,#2 \right>\right>}
\newcommand{\mx}[4]{\begin{bmatrix} #1 & #2 \\ #3 & #4 \end{bmatrix}}
\newcommand{\jac}[4]{\matrix{\dde{#1}{#3}}{\dde{#1}{#4}}{\dde{#2}{#3}}{\dde{#2}{#4}}} % u  v ~u ~v
\newcommand{\JAC}[4]{\matrix{\Dde{#1}{#3}}{\Dde{#1}{#4}}{\Dde{#2}{#3}}{\Dde{#2}{#4}}}
\newcommand{\caj}[4]{\matrix{\dde{#1}{#3}}{\dde{#2}{#3}}{\dde{#1}{#4}}{\dde{#2}{#4}}} % Transpose of above
\newcommand{\CAJ}[4]{\matrix{\Dde{#1}{#3}}{\Dde{#2}{#3}}{\Dde{#1}{#4}}{\Dde{#2}{#4}}}
\newcommand{\ev}[1]{\text{E} \left\lbrack #1 \right\rbrack}
\newcommand{\var}[1]{\text{Var} \left\lbrack #1 \right\rbrack}
\newcommand{\mgf}[1]{\ev{e^{t#1}}}
\DeclareMathOperator{\sech}{sech} % Why didn't these exist?
\DeclareMathOperator{\csch}{csch}

%%% SECTION TITLE APPEARANCE
\usepackage{sectsty}
\allsectionsfont{\sffamily\mdseries\upshape} % (See the fntguide.pdf for font help)
% (This matches ConTeXt defaults)

%%% ToC (table of contents) APPEARANCE
\usepackage[nottoc,notlof,notlot]{tocbibind} % Put the bibliography in the ToC
\usepackage[titles,subfigure]{tocloft} % Alter the style of the Table of Contents
\renewcommand{\cftsecfont}{\rmfamily\mdseries\upshape}
\renewcommand{\cftsecpagefont}{\rmfamily\mdseries\upshape} % No bold!

\usepackage{fancyhdr}
\pagestyle{fancy} % options: empty , plain , fancy
\renewcommand{\headrulewidth}{0.4pt}
\setlength{\headheight}{26pt}

\lhead{Derpnet \\ Technical Specification}
\chead{}
\rhead{[internal] \\ Revision 0}

\lfoot{}
\cfoot{\thepage}
\rfoot{}

%%%
%
% End header.
%
%%%

\title{Official Technical Specifications}
\author{The Derpnet Team}
%\date{}

\begin{document}
\maketitle

\section{Preamble}

This is the documentation of standards for the Derpnet secure communication
setup.  The latest version of this document can be found in our git
repository which is located \href{https://github.com/BUILDS-/Derpnet}{here}.

\subsection{Licensing}

All code in the project is licensed under the GNU General Public License
version 3.  A copy of this license can be found in the COPYING.txt file
within the repository, or obtained from the GNU project
\href{http://www.gnu.org/licenses/gpl.txt}{here}. 

\subsection{Declaration of Purpose}

The Derpnet project is designed with three goals in mind: security,
transparency, and usability.

\subsubsection{Security}

For us, ``security'' encompasses both security in the traditional sense as well
as anonymity.  Our idea is that the establishment of online identities
completely unassociated with their corresponding ``real-world'' counterparts
allows for ljadfglsdgbj

\subsubsection{Transparency}

Security through obscurity is no security at all.  A malicious attacker is
capable of reverse-engineering any component of a system, and generally is able
do so in a reasonably short amount of time.  Strong security relies on the
strength of its algorithms, not on the obscurity of said algorithms.  To that
end, we have chosen to take our development model to the extreme opposite, where
all our code is open to analysis by all.  In this way, we can benefit from the
expertise of the community while still maintaining a small, focused core team.

\subsubsection{Usability}



\subsection{Contributions and External Code}

\section{Architecture}

\subsection{Email Server}

\subsection{The IRC Protocol}

\subsection{OTR \&\& SSL Encryption}

\section{Unresolved Questions}

\section{Further Reading}

\section{Credits}

Derpnet is: $\alpha, \beta, \gamma, \delta$.

\end{document}
